\documentclass{beamer}

\usepackage[utf8]{inputenc}
\usepackage{amsmath}
\usepackage{amssymb}
\usepackage{tikz-cd}
\usepackage{mathtools}
\usepackage{multicol}
\usepackage{minted}
\usemintedstyle{borland}

\AtBeginSection[]
{
  \begin{frame}
    \frametitle{Table of Contents}
    \tableofcontents[currentsection]
  \end{frame}
}

%Information to be included in the title page:
\title{Property Testing with derived idempotents}
\author{Brian Zeligson}
\date{April 2020}



\begin{document}

\frame{\titlepage}

\begin{frame}
\frametitle{Table of Contents}
\tableofcontents
\end{frame}

\section{Challenges of Property Testing}

\begin{frame}
\frametitle{Challenges of Property Testing}

You don't know what your inputs are. \\ \pause
\medskip

Your properties are meant to hold over a broad set of inputs, they must be general. \\ \pause
\medskip

How do you make meaningful assertions without re-implementing the code under test? \\ \pause
\medskip

How is it handled at the source? \\
\medskip

\href{http://www.cse.chalmers.se/~rjmh/QuickCheck/manual.html}{QuickCheck} \\
\href{https://hypothesis.readthedocs.io/en/latest/}{Hypothesis} \\
\href{https://jsverify.github.io/}{JSVerify} \\

\end{frame}


\section{Isomorphisms: Ideal Property Testing Candidates}
\begin{frame}[fragile]
\frametitle{Isomorphism Defined}
\begin{tikzcd}
  & A
    \arrow[rr, bend left, swap, yshift = .7ex, swap, "to"]
  & & B
    \arrow[ll, bend left, yshift =  -.7ex, "from"]
\end{tikzcd}
\medskip
\begin{equation*}
\begin{aligned}
& \text{from(to(A))} = from \circ to = 1_A \\
& \text{to(from(B))} = to \circ from = 1_B
\end{aligned}
\end{equation*}
\end{frame}

\begin{frame}[fragile]
\frametitle{Isomorphism Example: encode {$\iff$} decode}
\begin{tikzcd}
  & JSON 
    \arrow[rr, bend left, swap, yshift = .7ex, swap, "encode"]
  & & String
    \arrow[ll, bend left, yshift =  -.7ex, "decode"]
\end{tikzcd}
\medskip
\begin{equation*}
\begin{aligned}
& \text{decode(encode(JSON))} = decode \circ encode = 1_{JSON} \\
& \text{encode(decode(String))} = encode \circ decode = 1_{String}
\end{aligned}
\end{equation*}
\end{frame}

\begin{frame}[fragile]
\frametitle{Isomorphism Example: reverse {$\iff$} reverse}
\begin{tikzcd}
  & String 
    \arrow[rr, bend left, swap, yshift = .7ex, swap, "reverse"]
  & & String 
    \arrow[ll, bend left, yshift =  -.7ex, "reverse"]
\end{tikzcd}
\medskip
\begin{equation*}
\begin{aligned}
& \text{reverse(reverse(String))} = reverse \circ reverse = 1_{String} \\
& \text{reverse(reverse(String))} = reverse \circ reverse = 1_{String} \\
\end{aligned}
\end{equation*}
\end{frame}

\section{Idempotents: Practical Property Testing Candidates}
\begin{frame}[fragile]
\frametitle{Idempotent Defined}
\begin{tikzcd}
  & A
    \arrow[rr, bend left, swap, yshift = .7ex, swap, "to"]
  & & B
    \arrow[ll, bend left, yshift =  -.7ex, "from"]
\end{tikzcd}
\medskip
\begin{equation*}
\begin{aligned}
& \text{to(from(B))} = to \circ from = 1_B \\ \pause
& \\
& \text{from(to(from(to(A))))} = \\ \pause
& from \circ to \circ from \circ to = \\ \pause
& from \circ (to \circ from) \circ to = \\ \pause
& from \circ 1_B \circ to = \\ \pause
& from \circ to \\
\end{aligned}
\end{equation*}
\end{frame}

\begin{frame}[fragile]
\frametitle{Idempotent Example: Run {$\xhookrightarrow{}$} Rest}
\begin{tikzcd}
  & Run 
    \arrow[rr, bend left, swap, yshift = .7ex, swap, "finish"]
  & & Rest 
    \arrow[ll, bend left, yshift =  -.7ex, "start"]
\end{tikzcd}
\medskip
\begin{equation*}
\begin{aligned}
& \text{start(finish(Run))} = start \circ finish = 1_{Run} \\ \pause
& \\
& \text{finish(start(finish(start(Rest))))} = \\
& finish \circ start \circ finish \circ start = \\
& finish \circ (start \circ finish) \circ start = \\
& finish \circ (1_{Run}) \circ start = \\
& finish \circ start \\
\end{aligned}
\end{equation*}
\end{frame}

\begin{frame}[fragile]
\frametitle{Idempotent Example: JSON {$\xhookrightarrow{}$} String}
\begin{tikzcd}
  & JSON 
    \arrow[rr, bend left, swap, yshift = .7ex, swap, "encode"]
  & & String
    \arrow[ll, bend left, yshift =  -.7ex, "decode"]
\end{tikzcd}
\medskip
\begin{equation*}
\begin{aligned}
& \text{decode(encode(JSON))} = decode \circ encode = 1_{JSON} \\ \pause
& \\
& \text{encode(decode(encode(decode(String))))} = \\
& encode \circ decode \circ encode \circ decode = \\
& encode \circ (decode \circ encode) \circ decode = \\
& encode \circ (1_{JSON}) \circ decode = \\
& encode \circ decode \\
\end{aligned}
\end{equation*}
\end{frame}

\section{Application: Deriving Idempotents for Property Testing}

\begin{frame}[fragile]
\frametitle{Making Properties Easy}

We know that properties are easy and effective when we have 
an isomorphism. \\
\medskip

\begin{minted}{haskell}
prop_RevRev xs = reverse (reverse xs) == xs
  where types = xs::[Int]
\end{minted}

\pause
\medskip
What about when we don't have an isomorphism? \\
\pause
\medskip

Can we \emph{find} an isomorphism?

\end{frame}

\begin{frame}[fragile]
\frametitle{Finding an isomorphism}

FizzBuzz does not belong to an isomorphism.
\medskip

\begin{minted}{scala}
object FizzBuzz {
  def apply(nums: List[Int]): List[String] =
    nums.map(n => (n, n % 3, n % 5) match {
      case (0, _, _) => "0"
      case (_, 0, 0) => "FizzBuzz"
      case (_, 0, _) => "Fizz"
      case (_, _, 0) => "Buzz"
      case _ => n.toString
    })
}
\end{minted}

\pause
\medskip
What is the closest isomorphism we can find? \\
\medskip

It helps to take a different perspective.

\end{frame}

\begin{frame}[fragile]
\frametitle{FizzBuzz as a Set Function}
\begin{tikzcd}[cramped, row sep = tiny]
  & 1 \arrow[r] & 1 \\
  & 2 \arrow[r] & 2 \\
  & 3 \arrow[r] & Fizz & ... \\
  ... & 5 \arrow[r] & Buzz & ... \\
  ... & 9 \arrow[ruu] &  & ... \\
  ... & 12 \arrow[ruuu] &  & ... \\
  ... & 15 \arrow[r] & FizzBuzz & ... \\
  ... & 25 \arrow[ruuuu] &  & ... \\
  ... & 30 \arrow[ruu] &  & ... \\
\end{tikzcd}
\medskip

We cannot have an isomorphism because inputs \emph{collapse} 
onto outputs. \\
This prevents construction of an inverse.
\end{frame}

\begin{frame}[fragile]
\frametitle{FizzBuzz as a Set Function, Partitioned Domain}
\begin{tikzcd}[cramped, row sep = tiny]
  & \{1\} \arrow[r] & 1 \\
  & \{2\} \arrow[r] & 2 \\
  & \{3, 6, 9, 12, ...\} \arrow[r] & Fizz \\
  & \{4\} \arrow[r] & 4 \\
  & \{5, 10, 20, 25, ...\} \arrow[r] & Buzz \\
  & \{7\} \arrow[r] & 7 & ... \\
  ... & \{15, 30, ...\} \arrow[r] & FizzBuzz \\
  & \{16\} \arrow[r] & 16 \\
\end{tikzcd}
\medskip

We have an isomorphism, can we fix the input type? \\ \pause
\medskip

With an idempotent.
\end{frame}

\begin{frame}[fragile]
\frametitle{FizzBuzz{$^{-1}$} as a Set Function, Idempotent}

We just pick one value from each input set.

\begin{tikzcd}[cramped, row sep = tiny]
  & 1 & \{1\} \arrow[l] & 1 \arrow[l] \\
  & 2 & \{2\} \arrow[l] & 2 \arrow[l] \\
  & 3 & \{3, 6, 9, 12, ...\} \arrow[l] & Fizz \arrow[l] \\
  & 4 & \{4\} \arrow[l] & 4 \arrow[l] \\
  & 5 & \{5, 10, 20, 25, ...\} \arrow[l] & Buzz \arrow[l] \\
  & 7 & \{7\} \arrow[l] & 7 \arrow[l] & ... \\
  ... & 15 & \{15, 30, ...\} \arrow[l] & FizzBuzz \arrow[l] \\
  & 16 & \{16\} \arrow[l] & 16 \arrow[l] \\
\end{tikzcd}
\medskip

This can be pre-composed with FizzBuzz to create an identity on the output set. \\
This means we have an idempotent on the input set. 

\end{frame}

\section{Live Coding: Property Testing FizzBuzz}

\end{document}


